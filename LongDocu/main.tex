%%% Research Diary - Entry
%%% Template by Mikhail Klassen, April 2013
%%% 
\documentclass[11pt,letterpaper]{article}

\newcommand{\workingDate}{\textsc{2022 $|$ March $|$ 14}}
\newcommand{\userName}{Natalie Kubicki}
\newcommand{\institution}{University of Cologne}
\usepackage{researchdiary_png}
% To add your univeristy logo to the upper right, simply
% upload a file named "logo.png" using the files menu above.


% Add packages
\usepackage{physics}
\usepackage{hyperref}
\usepackage{xcolor}
\usepackage{amsmath}
\usepackage{bm}
\usepackage{mathtools}

\begin{document}
\univlogo

\title{Research Diary - Example Entry}

{\Huge March Research on Non-Newtonian fluids}\\[5mm]
Done: Started to read book \textit{Cardiovascular mathematics} and different papers e.g.  \textit{Methods of Blood Flow Modelling} (see Notion-Files \newline \url{https://www.notion.so/22348433906041bc96a58526e35ab922?v=ce689bf670d9451c9c8a0eb71093342d} for full list of literature)

\subsection*{Subjects}

\begin{enumerate}
\item Familiarization with the properties and modeling of blood flow
\item Rheological models for blood - Research on modelling approaches for Non-Newtonian Fluids - In particular which model is suited for blood flow 
    \item Derivation of Navier-Stokes Equation - How is the assumption for Newtonian fluids integrated (linked with subject 2)
\end{enumerate}

\section{Familiarization with the properties and modeling of blood flow}
\subsection{Book Cardiovascular mathematics }
\subsubsection*{Properties of blood and cardiovascular system in general}

\textit{"Blood is in fact a suspension of cells and particles in plasma. A Newtonian constitutive equation is generally accepted as a good approximation of blood behaviour for large vessels. However, the study of circulation in smaller vessels and capillaries needs to abandon the Newtonian assumption for the fluid and account for the shear-thinning behaviour of blood."}(Preface) \newline\newline 
\textit{"The more it [blood] stirs the more it fluidifies (just think to the behaviour of tomato ketchup, another shear-thinning fluid). In other words, its (apparent) viscosity decreases with the increase of the rate of deformation. This effect is stronger in smaller vessels, like the arterioles, venules and the capillaries. Viscoelastic effects can be very important at the fine spatial scale (micro-circulation)[...]."}(Chapter 2)

\textbf{Shear-thinning behaviour}: In rheology, shear thinning is the non-Newtonian behavior of fluids whose viscosity decreases under shear strain. It is sometimes considered synonymous for pseudoplastic behaviour.

\textit{"The microcirculation is made up of three parts; the arterioles, the capillaries and the venules. These vessels are very small (5-30 $\mu$m in diameter) but very numerous so that local velocities are very small ($\approx$ 1mm/s). This means that the characteristic values of Re are very small so that viscous forces completely dominate any inertial forces in the flow. \textbf{As a result, virtually all of the resistance to flow is found in the microcirculation}. This is not to say that viscous effects are not important in the large vessels, the no-slip condition at the vessel walls ensures that viscosity is important in determining the detailed distribution of flow in the large vessels. However, it does mean  \textbf{that almost all of the pressure head losses in the circulation occur in the microcirculation}. Most of the pressure head losses actually occur in the arterioles. The arterioles have very thick muscular and highly ennervated walls; the ratio of wall tissues to lumen diameter is $\approx$1. The distribution of blood flow to different tissues is determined largely by these vessels whose resistance is highly dependent upon their diameter, which is controlled by contraction or relaxation of the smooth muscle in the wall."}(Chapter 1)

\textit{"The capillaries are the smallest vessels in the circulatory system and they are responsible for the bulk of exchange between the blood and the various tissues. Capillaries range from 5-8$\mu$m in diameter and 200-400$\mu$m in length. The bore of most capillaries is smaller than the largest dimension of the red blood cells, which means that the cells pass through the capillaries in single file and they must deform during their passage. \textbf{In the capillaries, therefore, blood can no longer be thought of as a homogeneous fluid and it is necessary to treat it as a multi-phase fluid composed of plasma and cells (particle flow)}."}(Chapter 1)

\textit{"There is a reduction in the effective viscosity of blood in the microcirculation (the Fahraeus-Lindquist effect) due to the steric exclusion of the blood cells from the wall regions of the vessels (the Fahraeus effect)."}(Chapter 1)

\textbf{Fahraeus-Lindquist effect}: \href{https://www.youtube.com/watch?v=IDrHCGcq-ak}{Short explanation video}; \href{https://www.combster.tv/t/1317657-fhruslindqvist-effect?utm_source=spectroom_web&utm_medium=banner&utm_campaign=cross_promo}{Also interesting video} (\href{https://www.youtube.com/watch?v=sWfCKdjxLc0}{Video turbulence}) \newline
is an effect where the viscosity of a fluid, in this case blood, changes with the diameter of the tube it travels through; in particular there's a decrease of viscosity as the tube's diameter decreases. This is because erythrocytes (red blood cells) move over the center of the vessel, leaving plasma at the wall of the vessel. \textit{("Below a critical vessel calibre (about 1mm), blood viscosity becomes dependent on the vessel radius and decreases very sharply. This is known as Fahraeus-Lindquist-effect. [..] red blood cells move to the central part of the capillary, whereas the plasma stays in contact with the vessel wall. This layer of plasma facilitates the movement of the red cells, thus causing a decrease of the apparent viscosity.")}(Chapter 2)

\textbf{Fahraeus effect}:  is the decrease in average concentration of red blood cells in human blood as the diameter of the glass tube in which it is flowing decreases. In other words, in blood vessels with diameters less than 500 micrometers, the hematocrit decreases with decreasing capillary diameter. The Fahraeus effect definitely influences the Fahraeus-Lindquist effect, which describes the dependence of apparent viscosity of blood on the capillary size, but the former is not the only cause of the latter.  \newline
Reason: The Fahraeus effect occurs because the average RBC (red blood cell ) velocity is higher than the average plasma velocity.

\subsubsection*{Mathematical Models}
\textit{"The mathematical equations of fluid dynamics are the key components of haemodynamics modelling. Rigorously speaking blood is not a fluid but a suspension of particles in the plasma, the latter being mainly made of water. Most important blood particles are red cells (erythrocytes), white cells (leukocytes), and platelets (thrombocytes). Being the most numerous, red cells are the main responsible for the special mechanical properties of blood. The prominent macroscopic effect of their presence is that blood is a shear-thinning, or thixotropic fluid."}(Chapter 2)


\textit{"Therefore, a first separation line between models for blood flow may be drawn: \textbf{on one side the Newtonian model which neglects shear thinning and viscoelastic effects and is suitable in larger vessels} or when we are not interested in the finer details of the flow, as non-Newtonian behaviour may affect, for instance, the size of the recirculation area behind a severe stenosis. On the other side, \textbf{in vessels of diameter, say, less than 1mm the use of Newtonian models is hardly justifiable. The small velocities and shear stress here involved call for the use of one of the non-Newtonian models} described in Chapter 6. Computationwise, non-Newtonian models which just modify the expression for the viscosity by making it dependent on the shear rate would increase the cost of computations of approximately 10 percent, because of the extra calculations and the increased non-linearity of the problem. Full visco-elastic models may instead be much more costly in terms of computing time."}(Chapter 2)

\paragraph*{Model flow in large
and medium sized vessels}
\textit{"Flow is here governed by the Navier-Stokes equations}
\begin{align}
    \frac{\partial \bm{u}}{\partial t} + \rho (\bm{u} \cdot \bm{\nabla})\bm{u} + \bm{\nabla}P - \bm{div}(\mu \bm{D}(\bm{u})) = \bm{f} \\
    \bm{div}(\bm{u}) = 0
\end{align}
\textit{"in a domain $\Omega \in \mathrm{}{R}^3$ representing the lumen of the vessel, or system of vessels, under investigation.
For a \textbf{Newtonian} fluid assumption the viscosity $\mu$ is kept constant.} 

\textit{"The principal unknowns are the velocity $\bm{u}$ and the pressure $P$, while the density $\rho$ is here constant. The term $\bm{f}$ in the right hand side accounts for the possible action of external forces, like gravity, and is often taken equal to zero in haemodynamics."}(Chapter 2)

\subsubsection*{Boundary conditions and Initial conditions} \textit{"The equations have to be supplemented with boundary conditions on $\partial  \Omega$.
We typically prescribe a velocity profile at the proximal boundary $\Gamma$ in, that is the section closest to the heart along the direction of the mean blood flow, which we will also denote as inflow boundary, even if the term inflow in not completely correct since in some major vessels we can have flow reversals. We then prescribe zero velocity at the fixed walls and the normal stresses $\bm{T} \cdot \bm{n}$ at the distal boundaries $\Gamma_{out}$  (also called outflow boundaries). Again, the term distal is meant with respect to the heart.\newline\newline
We need also to prescribe the initial status of the fluid velocity, for instance
$\bm{u}(\bm{x}, 0) = \bm{u}_0(\bm{x}) \bm{x} \in \Omega$,
being $\bm{u}_0$ a given quantity. We recall that $\bm{u}_0$  cannot be arbitrary, since it has to satisfy $\bm{div}\bm{u}_0  = 0$ to be admissible
! Unfortunately, in haemodynamics computations usually we do not know a
physically relevant initial condition. Therefore  $\bm{u}_0$ is usually chosen rather arbitrarily, often just equal to zero everywhere. It means that numerical computations may suffer a false transient linked to the incorrect initial data. If the boundary conditions are correct, however, it will decay quite rapidly.(...) A possibility to get a better guess of the initial data is to \textbf{solve a stationary Stokes problem}"}(Chapter 2) \newline\newline
\textit{"The situation is worsened when the compliance of the wall is taken into
account. The continuous exchange of energy between fluid and wall effectively makes the decay slower. In calculations of flow in compliant vessels it is normal practise to wait for at least three cardiac cycles before considering the influence of the initial data negligible"}(Chapter 2)
\subsubsection*{Properties of equations}\textit{"The solution of the Navier-Stokes equations may develop instabilities,
which are normally called turbulence. The responsible is the dynamics induced sure the importance of this term compared with the diffusive part given by by the non-linear convection term $ \rho (\bm{u} \cdot \bm{\nabla})\bm{u}$. It is therefore natural to measure the importance of this term compared with the diffusive part given by $\bm{div}(\mu \bm{D}(\bm{u}))$. This information is provided by the Reynolds number.  \textbf{If the Reynolds number is small, say at most of the order of 1000, the flow remains stable, and is called laminar}. In normal physiological situations, then, the values of the Reynolds number reached in the cardiovascular system do not allow the formation of full scale turbulence. Some flow instabilities may occur only at the exit of the aortic valve and limited to the systolic phase. In this region the Reynolds number may reach the value of few thousands only for the portion of the cardiac cycle corresponding to the peak systolic velocity, however, there is not enough time for a full turbulent flow to develop. \textbf{When departing from physiological conditions, there are several factors that may induce transition from laminar to turbulent flows. For instance, the increase of flow velocity because of physical exercise, or due to the presence of a stenotic artery or a prosthetic implant such as a shunt, may produce an increase of the Reynolds number and lead to localised turbulence. Smaller values of blood viscosity also raise the Reynolds number}; "}(Chapter 2)
\newline\newline \textit{"Knowing the velocity and the pressure fields allows the computation of the stresses, in particular the shear stresses to which an arterial wall is subjected  due to the blood movement. Wall shear stresses, are the force per unit area exerted by the
fluid tangentially to the wall. We have already mentioned their importance in relation with some vascular diseases, since endothelium cells react to shear stresses.Irregular, and in particular small or \textbf{oscillating shear stresses} \textit{[Wall shear stress is considered to be oscillating when its component along the main flow direction changes sign during the heart beat. In normal situations the component of wall shear stress along the main flow is always negative. Oscillating shear stresses are usually found in recirculation regions.]} may cause an alteration in the endothelium covering and induce inflammatory processes. Their calculation require a point-wise knowledge of the velocity and pressure field! To account for the compliance of the vessel wall we need to introduce
another unknown, namely the wall displacement $\bm{\eta}$"}(Chapter 2)
\subsubsection*{FSI} \textit{Smaller vessels experience a smaller relative movement than larger ones, where the change of radius during the heart beat may be of the order of 15 percent , like in the aorta. Therefore, the flow in the peripheral vessels, lets say more than two branching levels down from the aorta, can be reasonably modelled using a fixed geometry. \textbf{An exception} being the coronaries, whose movement is however dominated by the heart movement more than the fluid-structure interaction in the vessel. The effect of heart movement in the shear stress distribution in a coronary artery has been investigated. It has been found that it can be relevant, particularly in vessels with high curvature. Even in the larger vessels, at least in physiological situations, the main
characteristic of the flow are already captured by a fixed geometry model. \textbf{However, if more details are needed, such as a precise computation of shear stresses or the size of a recirculation region, then compliant models are better suited. Furthermore, if it is necessary to have an accurate description of pulse waves, for instance if one wants to investigate altered pressure pattern possibly caused by anomalous pulse wave reflections, like in the study of aortic aneurysms [285], then compliant models are mandatory}. The reason is that fixed geometry models simply cannot describe pulse waves: the propagation speed is here infinite because of the incompressible fluid. It is indeed the mechanical interaction between blood flow and vessel wall deformation that generates the pulse waves."}(Chapter 2)
\subsubsection*{General} \textit{A clear major feature of blood flow is its pulsatility. It may induce flow
reversal and recirculations near the arterial wall, a phenomenon that can have negative effects on the endothelium and stimulate the deposit of lipids and atherosclerosis. The latter effect is more likely to occur in specific vascular districts, like the carotid bifurcation"}(Chapter 2) \newline\newline \textit{With some approximation one may think that blood flow is periodic in time. Yet, this can be considered true only for relatively short periods, since the various human activities require to change the amount of blood sent to the various organs. Also the elastic properties of arteries (especially the arterioles) may vary depending on the request of blood by the peripheral organs. Indeed one of the aspects of current research in computational haemodynamics is the interaction between blood flow and the metabolic regulation. It presents several challenges from the mathematical modelling and numerical side. For the sake of space and because only partial results are available so far this aspect has not been extensively covered in this book (see Chapter 10). In several early studies, however, blood computations were made \textbf{using
steady flow. This can be considered acceptable in peripheral arteries, the capillary bed and in the veins,where the pulsatility of the flow is reduced thanks to the regularising effect of the compliance of the major arteries. In particular, micro-circulation is practically (but not completely) steady. The use of steady computations in larger vessels may again by justified by the lower computational cost.}}(Chapter 2)

\subsubsection*{(Temperature)} \textit{We mention that in some particular contexts, for instance in the hyperthermia treatment where some drugs are activated through an artificial localised increase in temperature (see [123, 219]), the variation of blood temperature may be relevant. Describing the evolution of temperature requires to introduce another partial differential equation which derives from the principle of energy conservation, and couple it with the Navier-Stokes equations. In large and medium sized vessel the coupling is weak, since here temperature variations have small influence in the flow field.[...] \textbf{Things are different in micro-circulation, where the combined effect of temperature on the blood apparent viscosity and on other mechanical properties of the vessel wall makes the situation more complex [178]. However, since in the physiological regime the temperature inside the human body is constant and the situations where temperature variations are relevant are rather special, we will not pursue this topic further in this book}\textbf{)}}(Chapter 2)

\subsubsection*{Mathematical Derivation} \textbf{ALE} \textit{As already mentioned in Chapter 2, in many cases of practical interest in haemodynamics, such as blood flowing in a compliant artery, the computa- tional domain for the fluid cannot be fixed in time, as it has to follow the displacements of the fluid-wall. Yet, the Lagrangian frame is not of help here, since certainly we do not wish to follow the evolution of the blood particles as they circulate along the whole cardiovascular system! We usually wish to compute the flow field in a domain confined in the area of interest, yet following the movement of the wall interface. The computational domain, which we will now indicate with $\omega$(t), is neither fixed nor a material subdomain, since its evolution is not governed by the fluid motion, but has to comply by that of the boundary $\partial\omega$(t), which is either given or the result of the coupling with a structural model. It is then necessary to introduce another, intermediate, frame of reference, called Arbitrary Lagrangian Eulerian (ALE).}

\subsubsection*{Navier-Stokes Equations and Boundary conditions} see Derivation of NS-Eqs

\subsubsection*{Chapter 4 is image processing}
\subsubsection*{Chapter 5 key flow paramaters} \textit{Reynolds number,$Re_D$, is overall the pre-eminent parameter, both in determining the stability of a flow and the persistence of geometric influences downstream of a bend or other disturbance. The reduced velocity, $U_{red}$ is a more appropriate parameter than the Womersley number for unsteady flows where there are significant streamwise flow variations. The Dean number De is the dominant parameter used to characterise flow in bends; in tightly curved or helical bends other parameters in addition to De may be required.}
\subsubsection*{Chapter 6 is rheology of blood - see section 2}

\newpage
\subsection{Paper A heterogeneous multi-scale model for blood flow}
\subsubsection*{Motivation for multiscale modeling}

\textit{"Blood flow on scales larger than 300$\mu$m is consistently modeled as a continuous fluid, as it is computationally more convenient because models no longer include the individual cell dynamics which greatly reduces the computational overhead.  {\color{blue}Continuous models either assume whole blood as a Newtonian fluid on larger scales or use a non-Newtonian blood viscosity model to approximate the departure of whole from the Newtonian description.  Non-Newtonian models describe the change in blood viscosity with a dependency either on shear rate like a power law fluid [22] or Carreau-Yasuda [2], or depends on yield stress like the Casson model [19]. \textbf{Since such models do not include the dynamics of the cells they may over estimate the transport behaviors which are a result of cell-cell collisions within whole blood suspensions}}. This may lead to an invalid description of particle diffusivities within whole blood."}

\textit{"In order to capture both the non-Newtonian viscosity change of whole blood along with the proper treatment of the transport of suspended blood cells a multi-scale model must be developed in order to correctly account for both processes on all scales of the cardiovascular system. (...) 
A benefit of such an HMM model applied to blood flow is that on the largest scales a continuous blood flow solver will be informed by a micro scale cell resolved blood flow solver, resolving the cell nature of whole blood by keeping computational overhead in mind."}
\newline\newline
\textbf{Very good schematic images about the method}
....










\newpage
\section{Rheological models for blood}
\subsection{Book Cardiovascular mathematics (Chapter 6)}
\textit{Rheology is the science of the deformation and flow of materials. It deals with the theoretical concepts of kinematics, conservation laws and constitutive relations, describing the interrelation between force, deformation and flow [...]  The object of haemorheology is the application of rheology to the study of flow properties of blood and its formed elements [...] \newline Therefore, the mathematical and numerical study of powerful, yet simple, constitutive models that can capture the rheological response of blood over a range of flow conditions is ultimately recognised as an important tool for clinical diagnosis and therapeutic planning}

\subsubsection*{Physical mechanisms behind the mechanical properties of blood}

\textit{Whole blood is a concentrated suspension of formed cellular elements that includes red blood cells (RBCs) or erythrocytes (40-45 percent), white blood cells (WBCs) or leukocytes (around 1 percent) and platelets or thrombocytes (55-60 percent).}

\textit{While plasma is nearly Newtonian in behaviour, whole blood exhibits marked non-Newtonian characteristics, particularly at low shear rates. The non-Newtonian behaviour of blood is mainly explained by three phenomena:\begin{enumerate}
    \item the erythrocytes tendency to form a three-dimensional microstructure at low shear rates
    \item their deformability
    \item their tendency to align with the flow field at high shear rates
\end{enumerate}
The formation and breakup of this 3D microstructure, as well as the elongation and recovery of red blood cells, contribute to bloods shear thinning, viscoelastic and thixotropic behaviour (here we refer to thixotropy as the dependence of the material properties on the time over which shear has been applied. This dependence is largely due to the finite time required for the three-dimensional structure of blood to form and break down.)}

\subsubsection*{Low shear rate behaviour: aggregation and disaggregation of erythrocytes} \textit{In the presence of fibrinogen and globulins (two plasma proteins), erythrocytes have the ability to form a primary aggregate structure of rod shaped stacks of individual cells called rouleaux. \textbf{At very low shear rates the rouleaux align themselves in an end-to-side and side-to-side fashion and form a secondary structure} consisting of branched three-dimensional aggregates [...] For blood at rest, the three-dimensional structure formed by the RBCs
appears \textbf{solid-like, appearing to resist flow until a finite level of force is applied} [...] \newline When blood begins to flow, the solid-like structure breaks into three-dimensional networks of various sizes which appear to move as individual units and reach an equilibrium size for a fixed shear rate. Increases in shear rate lead to a reduction in equilibrium size and lower effective viscosity[...] \newline \newline 
\textbf{The process of disaggregation under increasing shear is reversible}When the shear rate is quasi-statically stepped down to lower values, the individual cells form shorter chains, then longer rouleaux and eventually a 3-D microstructure \newline \newline 
{\color{blue}\textbf{The finite time necessary for equilibrium of the structure to be reached (both during aggregation and disaggregation) is responsible for the thixotropic behaviour of blood at low shear rates} \newline The associated time constants are a function of shear rate. The equilibria are found to be reached more rapidly at higher shear rates and more gradually with lower shear rates}}

\subsubsection*{High shear rate behaviour of whole blood: shear flow of dispersed erythrocytes}\textit{When blood is subjected to a constant shear rate ... the cells can be seen to rotate. With increasing shear rate, they rotate less and for shear rates above a value, they cease to rotate and remain aligned with the flow direction [...] (they) lose their biconcave shape, become fully elongated and are transformed into flat outstretched ellipsoids with major axes parallel to the flow direction. At this stage the collision of red cells only occurs when a more rapidly moving cell touches a slower one but there are no further interactions between the cells. \newline \newline 
{\color{blue}\textbf{The high deformability of erythrocytes is due to the absence of a nucleus, to the elastic and viscous properties of its membrane and also to geometric factors such as the shape, volume and membrane surface area}}}

\subsubsection*{Material blood}
\textit{Many of the continuum models for blood are examples of a large category of \textbf{constitutive models called incompressible simple fluids}. As defined in [100], ... an incompressible simple material . . . is a substance whose mass density never changes and for which the stress is determined, to within a pressure, by the history of strain. . . . We then define an incompressible simple fluid as an incompressible simple material with the property that all of its local configurations are intrinsically equivalent in response, with all observable differences in response being due to definite differences in history}\newline\newline
\textit{The mechanical response of incompressible Newtonian fluids in shear is completely determined by one material constant: the viscosity $\mu$. The response of general fluids is much more complicated and can include behaviours not displayed by Newtonian fluids such as rod climbing, shear thinning and memory. Remarkably, the behaviour of an arbitrary simple fluid in a broad class of flows called {\color{blue}\textbf{viscometric flows}} only requires knowledge of three material func- tions for that fluid. Appropriately, these three functions are called viscometric material functions and are intrinsic properties of the fluid}\newline

\textbf{Viscometric flow:} \textit{[...] that these are a special type of constant stretch history flow which, from the point of view of the fluid element, are indistinguishable from a steady simple shear flow described in terms of suitably chosen local Cartesian coordinates [...] 
three viscometric material functions are easily defined relative to
simple shear flow \begin{enumerate}
    \item Viscosity or Shear Viscosity
    \item First normal stress coefficient
    \item Second normal stress coefficient
\end{enumerate}}

\subsubsection*{Thixotropic response}\textit{
The formation of the three-dimensional microstructure and the alignment of the RBC are not instantaneous, which gives blood its thixotropic behaviour. [..] Outside of industrial applications, these definitions largely focus on the time dependence of rheological properties under fixed shear rate (e.g. viscosity, normal stress effects) arising from the finite time required for the breakdown and buildup in microstructure such as that just described for blood}


\subsubsection*{Constitutive models}
\textit{We will assume that all macroscopic length and time scales are sufficiently large compared to time and length scales at the level of the individual erythrocyte that the continuum hypothesis holds. \textbf{Thus the models presented in the pages that follow would not be appropriate in the capillary network}, for example, and for an overview of haemorheology in the microcirculation we refer the reader to the review articles of Popel and Johnson [393] and Pries and Secomb [394].[..]\newline it is important to consider in which flow regimes and clinical situations the non-Newtonian properties of blood will be important: (page 231) \newline \newline {\color{blue}  \textbf{Possible locations where the non-Newtonian behaviour will be significant include segments of the venous system and stable vortices downstream of some stenoses and in the sacs of some aneurysms}}. }\newline\newline
\textit{As a first step towards the macroscopic modelling of blood flow we recall the equations for the balance of linear momentum and conservation of mass (or incompressibility condition) for isothermal flow}
\begin{align}
    \frac{D \bm{u}}{D t} = -\bm{\nabla}P + \bm{div}(\bm{\tau})  \\
    \bm{div}(\bm{u}) = 0
\end{align}
\textit{Here, $\bm{\tau}$ denotes the extra-stress tensor accounting for differences in behaviour from a purely inviscid, incompressible fluid. To close the system of equations, we require an equation relating the state of stress to the kinematic variables such as rate of deformation of fluid elements. These constitutive equations and the elaboration of macroscopic constitutive models suitable for blood flow under certain flow conditions are the primary subjects of this section.}

\subsubsection*{Newtonian}\textit{ The simplest viscous fluid model is that due to Newton. On the assumption that the components of the extra-stress tensor are each linear isotropic functions of the components of the velocity gradient $\grad \bm{u} $, it may be shown that for an incompressible fluid
\begin{equation}
    \bm{\tau}= 2\mu \bm{D}(\bm{u}))
\end{equation}
this leads to the well-known Navier-Stokes equations for an incompressible viscous fluid. \textbf{This set of equations is commonly used with some justification to describe blood flow in the heart and healthy arteries} Blood is nonetheless non-Newtonian and in the previous sections evidence has been presented to show that under certain experimental or physiological conditions is inadequate as a constitutive relation for blood.}\newline\newline
\textit{ we first Discuss representative rheologically admissible constitutive equations with shear thinning viscosity, and then introduce the \textbf{Casson model}, a representative yield stress fluid}

\subsubsection*{Reiner-Rivlin-Fluids}
\textit{Without loss of generality (e.g. [15]), the most general incompressible constitutive model of the form $ \bm{\tau} = \bm{\tau}(\grad \bm{u})$ that respects invariance requirements
\begin{equation}
    \bm{\tau }= \phi_1(I_2, I_3) D(u) + \phi_2(I_2,I_3)D(u)^2
\end{equation}
where $I_2$ and $I_3$ are the second and third principal invariants of the rate of deformation tensor,
\begin{equation}
    I_2= \frac{1}{2}((tr(D(u))^2-tr(D(u)^2)), \hspace{5pt}  I_3 = det(D(u))
\end{equation}
and trace is identically zero for divergence free velocity fields essential for incompressible fluids (isochoric motions). Incompressible fluids of the form (6.16) are typically called Reiner-Rivlin fluids. The behaviour of Reiner-Rivlin fluids with non-zero values of $\phi_2$ in simple shear does not match experimental results on real fluids [15]. In addition, the dependence on the value of $I_3$ is often considered negligible [15].
}

\subsubsection*{Generalized Newtonian Fluids}
General form 
\begin{equation}
    \tau = 2 \mu(I_2) D(u)
\end{equation}
where $\mu$ is the same viscosity (a viscometric function) defined in (6.1). In viscometric flows, $I_3$ is identically zero and it is not necessary to explicitly assume the dependence of  $\mu$  on  $I_3$ is negligible. The quantity  $I_2$ is not a positive quantity, so it is useful to introduce a metric of the rate of deformation, denoted by $\Dot{\gamma} $
\begin{equation}
    \Dot{\gamma} = \sqrt{2 tr(D(u)^2)} = \sqrt{-4I_2} 
\end{equation}
the \textbf{generalised Newtonian model} takes then useful form, 
\begin{equation}
    \tau = 2 \mu(  \Dot{\gamma}) D(u)
\end{equation}

\subsubsection*{Power law model}\textit{A simple example of a generalised Newtonian fluid is that of the power-law fluid, which has viscosity function given by
\begin{equation}
    \mu = k \Dot{\gamma}^{n-1}
\end{equation}
k being a positive constant and n a constant chosen to have a maximum value 1, leading to a monotonic decreasing function of shear rate (shear thinning fluid) when $n< 1$ and a constant viscosity (Newtonian) fluid when $n =1$. \textbf{One of the major advantages of this model is that it is possible to obtain numerous analytical solutions to the governing equations. Two major drawbacks of the power-law model for the shear thinning case are that the zero shear rate behaviours are unphysical and limit the range of shear rates over which the viscosity is unbounded and the asymptotic limit as $\Dot{\gamma}\longrightarrow \infty$ is zero}. {\color{red}Both these behaviours are unphysical and limit the range of shear rates over which the power-law model is reasonable for blood}
}

\subsubsection*{Extension of
the power-law model from Walburn and Schneck}
\textit{
One of the more successful viscosity laws for blood is an extension of
the power-law model due to Walburn and Schneck [534]. In addition to the
shear rate, \textbf{they considered the dependence of the viscosity on the haematocrit
($H_t$) and total protein minus albumin (TPMA) content through the parameter k and n in (6.21). Using a nonlinear regression analysis they found that
shear rate and haematocrit were the two most important factors in decreasing
order of importance}. Based on these two factors, they formulated the following
expressions for k and n,\begin{equation}
k = C_1 exp(C_2H_t), \hspace{15pt} n=1-C_3H_t.
\end{equation}
They found an R-squared statistical increase from 62 percent to 88 percent when Ht was included in the power-law model in addition to shear rate. The statistical significance rose to nearly 91percent  when TPMA was also added. Walburn and Schneck attribute the two parameter model (6.21) to Sacks [441].
}

\subsubsection*{Quemada model} \textit{In 1978 Quemada [412] used a semi-phenomenological approach to develop a constitutive law suitable for concentrated disperse systems (such as blood) that had an apparent viscosity $\mu$ determined from
\begin{equation}
    \mu = \mu_f (1 -\frac{1}{2}\frac{k_0+k_{\infty}\sqrt{\Dot{\gamma}/\Dot{\gamma_C}}}{1+\sqrt{\Dot{\gamma}/\Dot{\gamma_C}}}\phi)^{-2}
\end{equation}
where $\mu_F, \phi \hspace{4pt }and \Dot{\gamma_C}$ are the viscosity of the suspending fluid, the volume concentration of the dispersed phase and a critical shear rate.
}



\subsubsection*{Cassons equation}
\begin{equation}
    \sqrt{\abs{\tau_{12}}}= \sqrt{K}\sqrt{\Dot{\gamma}}+\sqrt{\abs{\gamma_0}}
\end{equation}
for the absolute value of the shear stress $\abs{\tau_{12}}$ as a function of the shear rate when the magnitude of the shear stress exceeds that of a yield stress $\gamma_0$

\textit{The controversy over the existence of a yield stress and the use of it as
a material parameter were introduced in Section 6.3.4. Here, we briefly sum- marise some of the results obtained for these measurements, but caution that measurements of the yield stress are expected to be quite sensitive to the microstructure of the blood prior to yielding, which is in turn expected to be sensitive to both the shear rate history as well as time [338] (...) and confirmed the importance of the presence of fibrinogen for the magnitude of the yield stress. Also haematocrit levels had to exceed a criti- cal threshold (typically between 0.05 and 0.08) for there to be a measurable yield stress. Results in the literature for the yield stress of blood show it to be very small, however:(..)}

\subsubsection*{Material parameters for blood in generalised Newtonian andCassonmodels}
see book
\textit{As discussed earlier in this chapter, the material parameters of blood are
quite sensitive to the state of blood constituents as well as temperature. The dependence on temperature has been found to be similar to water}

\textit{Table 6.2 summarises some of the most common generalised Newtonian
models that have been considered in the literature for the shear dependent viscosity of whole human blood.}

\textit{As discussed earlier in this chapter, the material parameters of blood are
quite sensitive to the state of blood constituents as well as temperature. The dependence on temperature has been found to be similar to water}

\subsubsection*{Viscoelastic and thixotropic models}
 \textit{Experimental in vitro evidence for the viscoelastic behaviour of human blood and discussion of its connection with the storage and dissipation of energy during the distortion of the 3D microstrucure formed by the RBC at low shear rates abounds in the literature:}

\textit{\textbf{A word of caution is in order at this point, however. A study of blood in sinusoidal flows in glass tubes by Federspiel and Cokelet in 1984 [143] using tube diameters, flow rate amplitudes and oscillation frequencies that attempted to mimic those in small arteries indicated that blood elasticity was effectively negligible in this flow regime. Differences with measurements made earlier by Thurston [503] were attributed to the larger shear rates in Federspiel and Cokelets work, and Thurstons work was suggested as being more applicable to venous flow or pathological low-flow rate flows than arterial flow}}

\textit{In addition to being a viscoelastic fluid, \textbf{the fact that red cell aggregates neither form nor break up instantaneously leads to blood being thixotropic }(see, Section 6.3.3) and the reader is also referred to [215,231,505], for example, for further discussion}


\textit{None of the above models iaccounts for either the viscoelasticity or the thixotropy of blood}



\textit{Viscoelastic constitutive models of differential type, suitable for describing blood, have been proposed recently by Yeleswarapu [548, 549] and by Anand and Rajagopal [10] (the latter being developed in the context of the general thermodynamic frame- work of Rajagopal and Srinivasa [417])}


\textit{The common outcome of the modelling done by all these authors is a \textbf{generalised Maxwell-type equation for the stress due to the size and position distributions of the rouleaux}. Both viscosity and relaxation time are functions of a structure variable which, in the papers cited above, is either the number fraction of red blood cells in aggregates (more generally, aggregated particles)  or of aggregated cell faces [5}

\textit{
In 2006 Owens [141, 367] followed ideas drawn from the classical theory
of network models for viscoelastic fluids to derive a relatively simple single-mode structure-dependent generalised Maxwell model for the contribution $\bm{\tau}$
of the erythrocytes to the total Cauchy stress. In the model developed in [141,
367] the erythrocytes were represented in their capacity to be transported,
stretched and orientated in a flow by Hookean dumbbells, thus limiting the
model to low shear rate flows. The extra-stress tensor $\bm{\tau}$ in Eq. (6.14) may be
written as the sum of a Newtonian viscous stress tensor and an elastic stress
tensor $\bm{\tau_E}$ (represents the contri- bution to the extra-stress due to the erythrocytes.) }\begin{equation}
    \bm{\tau}= 2\mu \bm{D}(\bm{u})+\bm{\tau_E}
\end{equation}

{\color{red} \subsubsection*{Comparison of predictions of constitutive models with experimental data}}
\textit{In a study in 1980 by Easthope and Brooks (...) . The model of Walburn and Schneck resulted in the closest fit. The predictions of the Walburn and Schneck model have been compared with those of a Newtonian fluid, Casson model and Bingham model for laminar flow through a straight tube under flow conditions bearing some similarity to those that exist in the femoral artery by Rodkiewicz et al (..) . The Walburn and Schneck model was seen to give markedly different results from the other models in pulsatile flow and these were stated as being in conformity with some experimental results [289]. The authors noted that the constitutive model of Walburn and Schneck was developed for low shear rates, however, and was not valid for certain shear rate regimes seen in their pulsatile flow simulations}

\textit{Details of a recent comparison between a Newtonian, Casson, power-law
and Quemada [412] model are to be found in the paper of Neofytou [348]. The author considered the case of channel flow where part of one of the channel walls was forced to oscillate laterally, this being claimed to reproduce some flow phenomena seen under realistic arterial conditions. The Casson and Quemada models were seen to agree well in their predictions and were preferred over the power-law model which has an unbounded viscosity at zero shear rate}


\textit{A full description of all the comparisons that have been performed with
the model of Owens may be found in [141]. Other viscoelastic models have already been used with some success for the simulation of this flow. For exam- ple, the pressure field predictions of the viscoelastic model of Anand and Rajagopal [10] were in reasonable agreement with the data of Thurston [504] for oscillatory tube flow and comparisons were also made between the experi- mental data and the results from a model of Yeleswarapu [548,549] and gener- alised Oldroyd-B and Maxwell models. Neither of the generalised models was found to give satisfactory results for oscillatory flows, however.}

\subsubsection*{Conclusion}

\textit{the mate- rial properties of flowing human blood, and in particular its shear viscosity, elasticity and thixotropy may be explained in terms of the complex evolving microstructure, and especially that of the deforming and migrating red blood cells in their different states of aggregation. We would suggest, therefore, that the most promising rheological models to date are those developed from an underlying microstructure similar to that of blood (albeit necessarily simpli- fied). The retention of sufficient detail at the microscopic level may be hoped to translate into faithful reproduction of some of the complex characteristics of blood, particularly those associated with its thixotropic nature}

\textit{Secondly, when we write of the desirability of sufficient detail at the microstructural level being retained in rheological models we mean just sufficient. Although we want to be able to successfully predict non-Newtonian effects in, say, an aneurysm we would like any reasonable model to collapse to the Navier-Stokes equations (with some suitable apparent viscosity) in bulk arterial flow, for example. A model that is unnecessarily complicated or costly, may have a sound rheological foundation but has little chance of attracting the attention of the medical community and therefore of being implemented in practical situations}

\textit{Thirdly, and finally, the development of stable, accurate and affordable numerical methods tailored to the new set of constitutive equations for blood is of the utmost importance. For example, proper account must be taken of the mathematical type of the system of equations, and the possible addition of elastic stress variables make the use of parallelisable algorithms even more crucial than they are in present day CFD Newtonian solvers. Faster large-scale computing platforms are open- ing up new possibilities in simulation and visualisation.}


\newpage
\subsection{Review Paper A REVIEW ON RHEOLOGY OF NON-NEWTONIAN PROPERTIES OF BLOOD}
\subsubsection*{INTRODUCTION}
\textit{Most commonly, the viscosity (the measure of a fluid's ability to resist gradual deformation by shear or tensile stresses) of non-Newtonian fluids is dependent on the shear rate or shear rate history. Some non-Newtonian fluids with shear- independent viscosity, however, still exhibit normal stress-differences or other non- Newtonian behavior. [...] \newline\newline Blood is a complex fluid with non-Newtonian characteristics, it has a shear-thinning behavior [3] and often exhibits a  \textbf{yield stress (viscoplasticity)} [4-6] with  \textbf{potential history effects (thixotropy) }[7]. [...]The rheological complexity of blood is attributed to its constituents. Rheologically, blood is primarily characterized as a concentrated suspension of elastic, deformable red blood cells (RBCs). However, it also contains other ingredients such as leukocytes and platelets within plasma. \textbf{Yield stress is an important characteristic of blood rheology and an essential component of its non-Newtonian nature}. Experimental evidence for its association with blood has been provided in many investigations. [...] \newline  \newline This may also explain a possible controversy about the thixotropic nature of blood [13] as the thixotropic-like behavior may be explained by other non-Newtonian characteristics of blood. Thixotropy is more pronounced at low shear rates with a long-time scale. The effect, however, seems to have a less important role in blood than other non-Newtonian effects such as shear thinning [14], and this could partly explain the limited amount of studies dedicated to this property. The thixotropic behavior of blood is very sensitive to blood composition and hence it can demonstrate big variations between different individuals and under different biological conditions. }

\subsubsection*{RHEOLOGY OF BLOOD}

\textit{Blood behaves like a non-Newtonian fluid whose viscosity varies with shear rate. The non-Newtonian characteristics comes from the presence of various cells in the blood that make blood a suspension of particles. When the blood begins to move, these particles (or cells) interact with plasma and among themselves. Hemorheologic parameters of blood include whole blood viscosity, plasma viscosity, red cell aggregation, and red cell deformability (or rigidity). From a biological point of view, blood can be considered as a tissue comprising various types of cells (i.e., RBCs, WBCs, and platelets) and a liquid intercellular material (i.e., plasma). From a rheological point of view, blood can be thought of as a two-phase liquid; it can also be considered as a solid-liquid suspension, with the cellular elements being the solid phase.}

\subsubsection*{DETERMINANTS OF BLOOD VISCOSITY }
\textit{Theory that completely accounts for
the viscous properties of blood, and some of the key determinants have been identified: The \textbf{four main determinants of whole blood viscosity are (1) plasma viscosity, (2) hematocrit, (3) RBC deformability and aggregation, and (4) temperature}. The first three factors are parameters of physiological concern because they pertain to changes in whole blood viscosity in the body. {\color{blue}\textbf{The second and third factors, hematocrit and RBC (red blood cells) aggregations, are the main contributors to the non-Newtonian characteristics of shear-thinning viscosity and yield stress.}}}



\subsubsection*{YIELD STRESS} \textit{In addition to non-Newtonian viscosity, blood also exhibits a yield stress. The source
of the yield stress is the presence of cells in the blood, particularly red cells. When such a huge amount of red cells of 8-10 microns in diameter is suspended in plasma, cohesive forces among the cells are not negligible. \textbf{The forces existing between particles are van der Waals-London forces and Coulombic forces . Hence, in order to initiate a flow from rest, one needs to have a force that is large enough to break up the particle-particle links among the cells}. However, blood contains red cells and still moves relatively easily. The healthy red cells behave like liquid drops because the membranes of red cells are so elastic and flexible. Note that in a fluid with no suspended particles, the fluid starts to move as soon as an infinitesimally small amount of force is applied. \textbf{Such a fluid is called a fluid without yield stress. Examples of fluid with no yield stress include water}. Examples of fluids having yield stress include blood, ketchup, salad dressings, grease, paint, and cosmetic liquids.} \newline\newline
\textit{The magnitude of the yield stress of human blood (...) is almost independent of temperature in the range of 10-37 degree} \newline\newline
\textit{Blood shows a Newtonian fluids character when it flows through larger diameter arteries at high shear rates, but it exhibits a remarkable non-Newtonian behavior when it flows through small diameter arteries at low shear rates. Moreover, there is an increase in viscosity of blood at low rates of shear as the red blood cells tend to aggregate into the Rouleaux form. \textbf{Rouleaux form behaves as a semi-solid along the center forming a plug flow region. In the plug flow region, we have a flattened parabolic velocity profile rather than the parabolic velocity profile of a Newtonian fluid. This behavior can be modeled by the concept of yield stress.} } 



\subsubsection*{\color{red}Casson fluid model} \textit{is a non-Newtonian fluid with yield stress, which is widely used for modeling blood flow in narrow arteries. Many researchers have used the Casson fluid model for mathematical modeling of blood flow in narrow arteries at low shear rates. It has been demonstrated by Blair [28] and Copley [29] that the Casson fluid model is adequate for the representation of the simple shear behavior of blood in narrow arteries. Casson examined the validity of the Casson fluid model in his studies pertaining to the flow characteristics of blood and reported that at low shear rates, the yield stress for blood is nonzero}

\textit{The yield stress characteristic of blood seems to vanish or become negligible when hematocrit level falls below a critical value. Yield stress contributes to the blood clotting following injuries and subsequent healing, and may also contribute to the formation of blood clots (thrombosis) and vessel blockage in some pathological cases such as strokes. The magnitude of yield stress and its effect could be aggravated by certain diseased states related to the rheology of blood, like polycythemia vera, or the structure of blood vessels such as stenosis.}

\subsubsection*{THIXOTROPY }\textit{The phenomenon of thixotropy in a liquid result from the microstructure of the liquid
system. Thixotropy may be explained as a consequence of aggregation of suspended particles. If the suspension is at rest, the particle aggregation can form, whereas if the suspension is sheared, the weak physical bonds among particles are ruptured, and the network among them breaks down into separate aggregates that can disintegrate further into smaller fragments [40]. After some time at a given shear rate, a dynamic equilibrium is established between aggregate destruction and growth, and at higher shear rates, the equilibrium is shifted in the direction of greater dispersion. The relatively long time required for the microstructure to stabilize following a rapid change in the rate of flow makes blood thixotropy readily observable [...] At high shear rates, structural change occurs more rapidly than at low shear rates.}

\subsubsection*{BLOOD VISCOSITY MEASUREMENT} \textit{Blood viscosity is mainly determined by the hematocrit and varies with the shear rate as a non-Newtonian fluid [44,} \newline\newline
\textit{Lee et al. [54] studied the applicability of \textbf{two non-Newtonian constitutive models ({\color{red}Casson} and {\color{red}Herschel-Bulkley models}) in the determination of the blood viscosity and yield stress using a pressure-scanning microfluidic hemorheometer}. The present results were compared with the measurements through a precision rheometer (ARES2). For a Newtonian fluid (standard oil), the two constitutive models showed excellent agreement with a reference value and the measurement of ARES2. For human blood as a non-Newtonian fluid, both the Casson and Herschel-Bulkley models exhibited similar viscosity results over a range of shear rates and showed excellent agreement with the ARES2 results. The Herschel-Bulkley model yielded a slightly higher value than other results at low shear rates, which may be due to the relatively high value of the yield stress. The yield stress values for whole blood were 14.4 mPa for the Casson model and 32.5 mPa for the Herschel-Bulkley model, respectively. {\color{blue}\textbf{Thus, their study showed that the {\color{red}Casson model} would be better than the {\color{red}Herschel-Bulkley model} for representing the non-Newtonian characteristics of blood viscosity [54]}}.}

\textit{Kang et al. presented the first experimental work on the viscosity measurement of adult zebrafish whole blood using a capillary pressure-driven microfluidic viscometer. After the device calibration with water, the viscosity measurement of human whole blood was performed and in good agreement with published data, demonstrating the reliability of the device. \textbf{{\color{red}Power law} and {\color{red}Carreau-Yasuda rheological} models were used to model the non-Newtonian behaviors of the human }and zebrafish blood [56].}


\textit{This result indicates that blood viscosity is likely to be more affected by blood cell particle numbers than by blood cell volume. Although this comparison has not been addressed in human clinical studies, a previous animal study demonstrated that the blood cell count had a stronger influence on blood viscosity than the hematocrit [57].}

\subsubsection*{HEAT TRANSFER OF BLOOD FLOW}
\textit{The effects of flow parameters namely Grashof number (Gr), Prandtl number (Pr), heat source parameter (N), Hartmann number (M) and decay parameter ($\lambda$) on the velocity and heat functions have been observed}
\subsubsection*{CONCLUSION} \textit{In this paper, rheological characteristics of blood were studied. As referred to in previous work, blood is a complex fluid with non-Newtonian characteristics. It often represents a yield stress (viscoplasticity) and has a shear-thinning behavior and due to the pulsative nature of the blood flow, it shows a thixotropy treatment. Blood has four main determinants including plasma viscosity, hematocrit, red blood cell deformability, and temperature. Blood shows a Newtonian fluids character when it flows through larger diameter arteries at high shear rates, but it exhibits a remarkable non-Newtonian behavior when it flows through small diameter arteries at low shear rates. Despite the fact that thixotropy is a transient property, due to the pulsative nature of the blood flow the thixotropic effects may have long term impact on the blood circulation. It should be remarked that time dependent effects whether thixotropic or viscoelastic in nature, should be expected in blood flow due to the pulsatility of blood flow and the rapid change in the deformation conditions during blood circulation. Modeling of heat transfer in the body can be used in surgery, especially open-heart surgery, as well as in making artificial blood vessels.}
\newpage
\subsection{Paper Methods of Blood Flow Modelling}
\subsubsection*{Hemorheology}
\textit{ the science of deformation and flow of blood and its formed elements. This field includes investigations of both macroscopic blood properties using rheometric experiments as well as microscopic properties in vitro and in vivo. }
\subsubsection*{Blood components} 
\textit{Blood is a concentrated suspension of several formed cellular elements, red blood cells (RBCs or erythro-cytes), white blood cells (WBCs or leukocytes) and platelets (thrombocytes), in an aqueous polymeric and ionic solution, the plasma, composed of  water and particles, namely, electrolytes, organic
molecules, numerous proteins (albumin, globulins and fibrinogen) and waste products.\newline [...] 
Normal erythrocytes are biconcave discs with a mean
diameter of 6 to 8 $\mu$m and a maximal thickness of 1.9 $\mu$m.}


\subsubsection*{Non-Newtonian properties of blood} \textit{The mechanical properties of blood should be studied by considering a fluid containing a suspension of particles. A fluid is said to be Newtonian if it satisfies the Newtons law of viscosity (the shear stress is proportional to the rate of shear and the viscosity is the constant of proportionality). Blood plasma, which consists mostly of water, is a Newtonian fluid. However, the whole blood has complex mechanical properties which become particularly significant when the particles size is much larger, or at least comparable, with the lumen size. \textbf{In this case, which happens at the microcirculation level (in the small arterioles and capillaries) blood cannot be modelled has a homogeneous fluid and it is essential to consider it as a suspension of blood cells (specially RBCs) in plasma.}}

\textit{Otherwise, depending on the size of the blood vessels and the flow behaviour, it is approximated as a
Navier-Stokes fluid or as a non-Newtonian fluid.{\color{blue}\textbf{ Here we assume that all macroscopic length and time scales are sufficiently large compared to length and time scales at the level of the individual erythrocyte so that the continuum hypothesis holds.}} }

\subsubsection*{Viscosity of blood}
\textit{\textbf{In general blood has higher viscosity than plasma, and when the hematocrit rises, the viscosity of the suspension increases and the non-Newtonian behaviour of blood becomes more relevant, in particular at {\color{blue}very low shear rates}}. The apparent viscosity increases slowly until a shear rate less than 1 s-1
where it rises markedly. The reason for this is that at low shear rates the erythrocytes have the ability to form a primary aggregate structure of rod shaped stacks of individual cells called \textbf{rouleaux}, that align to each other and form a secondary structure consisting of branched three-dimensional (3D) aggregates [118]. It has been experimentally observed that rouleaux will not form if the erythrocytes have been hardened or in the absence of fibrinogen and globulins (plasma proteins) [30]. (In fact, suspensions of erythrocytes in plasma demonstrate a strong non-Newtonian behaviour whereas when they are in suspension in physiological saline (with no fibrinogen or globulins) the behaviour of the fluid is Newtonian). For standing blood subjected to a shear stress lower than a critical value, these 3D structures can form and blood exhibits yield stress and resists to flow until a certain force is applied. This can happen only if the hematocrit is high enough.}

\textit{\textbf{{\color{blue}At moderate to high shear rates}}, RBCs are dispersed in the plasma and the properties of the blood
are influenced by their tendency to align and form layers in the flow, as well as to their deformation. The effect of RBC deformability on the viscosity of suspensions was clearly shown in}

\textit{\textbf{{\color{blue}For shear rates above 400 s-1}}, the RBCs lose their biconcave shape, become fully elongated and are transformed into ellipsoids with major axes parallel to the flow direction. The tumbling of the RBCs is absent, there are almost no collisions, and their contours change according to the tank-trading motion of the cells membranes about their interior. The apparent viscosity decreases and this becomes more evident in smaller than in larger vessels. This happens with vessels of internal diameter less than 1 mm and it is even more pronounced in vessels with a diameter of 100 to 200 $\mu$m.The geometric packing effects and radial migration of RBCs can act to lower the hematocrit adjacent to the vessel wall and contribute to decrease the blood viscosity. This is known as the \textbf{Fahraeus-Lindqvist effect}. \textbf{Plasma skimming} is another effect that results in diminishing the viscosity when blood flows into small lateral vessels compared with the parent vessel.}

\textbf{Plasma skimming:} The natural separation of red blood cells from plasma at bifurcations in the vascular tree, dividing the blood into relatively concentrated and relatively dilute streams.

\textbf{\textit{As a consequence of this behaviour we can say that one of the non-Newtonian characteristics of blood
is the shear thinning viscosity. This happens in small size vessels or in regions of stable recirculation, like in the venous system and parts of the arterial vasculature where geometry has been altered and RBC aggregates become more stable, like downstream a stenosis or inside a saccular aneurysm. However, in most parts of the arterial system, blood flow is Newtonian in normal physiological conditions}}

\subsubsection*{Constitutive Models}

\textit{{\color{red}Simplest constitutive
model for \textbf{incompressible viscous fluids based on the assumption that the extra stress tensor is proportional to the symmetric part of the velocity gradient}}}\begin{equation}
    \bm{\tau}= 2\mu \bm{D}(\bm{u}))
\end{equation}
with rate of deformation tensor \begin{equation}
    D_{ij} = \frac{1}{2} \large( \frac{\partial u_i}{\partial x_j} + \frac{\partial u_j}{\partial x_i} \large) \hspace{10pt} i,j=1,...3
\end{equation}
\textit{The substitution of $\bm{\tau}$ in the equations of the conservation of linear momentum and mass (or incompressibility condition) for isothermal flows given by well-known Navier-Stokes equations for an incompressible viscous fluid}

\textit{As already discussed, this set of equations is commonly used to describe blood flow in healthy arteries.
However, under certain experimental or physiological conditions, particularly at low shear rates, blood exhibits relevant non-Newtonian characteristics and more complex constitutive models need to be used.}

\textit{In this case, we require a more general constitutive equation relating the state of stress to the rate of deformation which satisfies invariance requirementts.  One of the simplest is the special class of {\color{blue}\textbf{Reiner-Rivlin fluids, called generalised Newtonian fluids}}, for which}
\begin{equation}
    \bm{\tau}= 2\mu(\Dot{\gamma}) \bm{D}(\bm{u}))
\end{equation}
\textit{where $mu(\Dot{\gamma})$ is a shear dependent function and $\Dot{\gamma}$ is the shear rate (a measure of the rate of deformation) defined by}
\begin{equation}
    \Dot{\gamma}=\sqrt{2 tr (\bm{D}(\bm{u}))^2)} = \sqrt{-4 \Pi_D}
\end{equation}
Here $\Pi_D$ denotes the second principal invariant of the tensor $\bm{D}$, given by
\begin{equation}
   \Pi_D = \frac{1}{2} ((tr\bm{D})^2 -tr \bm{D}^2)
\end{equation}
\textbf{A simple example of a generalised Newtonian fluid is the {\color{red}power-law fluid}, for which the viscosity function is given by}
\begin{equation}
   \mu(\Dot{\gamma}) = K \Dot{\gamma}^{n-1}
\end{equation}
\textit{the positive constantts n and K being the power-law index and the consistency, respectively. This model includes\begin{enumerate}
    \item as a particular case, the constant viscosity fluid (Newtonian) when $n = 1$.
    \item For $n < 1$ it leads to a monotonic decreasing function of the shear rate (shear thinning fluid) 
    \item for n > 1 the viscosity increases with shear rate (shear thickening fluid).
\end{enumerate}  The shear thinning power-law model is often used for blood, due to the analytical solutions easily obtained for its governing equations, but it predicts an unbounded viscosity at zero shear rate and zero viscosity when $\Dot{\gamma} \longrightarrow  \infty$, which is unphysical}

\textit{{\color{red}\textbf{Extensions of the power-law model is due to Walburn and Schneck}} who considered
the dependence of the viscosity on the hematocrit ($H_t$) and total protein minus albumin (TPMA) in the
constants n and K, based on nonlinear regression analysis}
\begin{equation}
    K = C_1 exp(C_2 H_t), \hspace{20pt}  n = 1- C_3 H_t
\end{equation}
Comparison between different models see original paper and cited paper!

\textit{Viscosity functions with bounded and non-zero limiting values of viscosity can be written in the general
form or
 \begin{equation}
     \mu(\Dot{\gamma}) = \mu_{\infty} + (\mu_0 - \mu{\infty})F(\Dot{\gamma})
 \end{equation}
 , in non-dimensional form as
 \begin{equation}
     \frac{\mu(\Dot{\gamma})- \mu_{\infty} }{\mu_0 - \mu{\infty}}= F(\Dot{\gamma})
 \end{equation}
Here, $(\mu_0$ and $\mu{\infty})$ are the asymptotic viscosity values at zero and infinite shear rates and $F(\Dot{\gamma})$ is a shear dependent function, satisfying the natural limit condition, i.e. goes to 1 for strain rate to zero and goes to 0 if strain rate goes to infinity}

{\color{red}\textit{\textbf{Different choices of the function $F(\Dot{\gamma})$  correspond to different models for blood flow, with material constants quite sensitive and depending on a number of factors including hematocrit, temperature, plasma viscosity, age of RBCs, exercise level, gender or disease state.}}}


\begin{table}[htbp]
\centering
\begin{tabular}{ccc}
\toprule
Model & $F(\Dot{\gamma})$  & Material constants for blood \\
\midrule
\textbf{Powell-Eyring} & $\frac{sinh{^-1}(\lambda\Dot{\gamma})}{\lambda\Dot{\gamma}}$ & $\lambda=5.383s$\\
\textbf{Cross}  & $\frac{1}{1+(\lambda\Dot{\gamma})^m}$ & $\lambda=1.007s, m = 1.028$\\
\textbf{Modified Cross}  & $\frac{1}{1+((\lambda\Dot{\gamma})^m)^a}$ & $\lambda=3.736s, m = 2.406, a =0.254$ \\
\textbf{Carreau}  &$(1+(\lambda\Dot{\gamma})^2)^{\frac{n-1}{2}}$ & $\lambda=3.313s, n= 0.3568$\\
\textbf{Carreau-Yasuda}  & $(1+(\lambda\Dot{\gamma})^a)^{\frac{n-1}{a}}$ & $\lambda=1.902s, n= 0.22, a=1.25$\\
\bottomrule
\end{tabular}
\caption[Tabelle]{Material constants for various generalised Newtonian models for blood with $\mu_0$ = 0.056Pa.s, $\mu_{\infty}$ = 0.00345Pa.s}
\label{tab:toll1}
\end{table}



\subsubsection*{Viscoelasticity and thixotropy of blood Viscoelastic}

\textit{\textbf{Viscoelastic fluids} are viscous fluids which have the ability to store and release energy. The viscoelasticity of blood at normal hematocrits is primarily attributed to the reversible deformation of the RBCs 3D microstructures [131]. \textbf{Elastic energy is due to the properties of the RBC membrane which exhibits stress relaxation }[42] and the bridging mechanisms within the 3D structure. Moreover, the experimental results of Thurston \textbf{imply that the relaxation time depends on the shear rate}. The reader is referred to [131] for a review of the dependence of blood viscoelasticity on factors such as temperature, hematocrit and RBC properties. {\color{blue}\textbf{In view of the available experimental evidence, it is reasonable to develop non-Newtonian fluid models for blood that are capable of shear thinning and stress relaxation, with the relaxation time depending on the shear rate. To date, very little is known concerning the response of such fluids}}. In fact, viscoelastic properties are of relatively small magnitude and they have generally only been measured in the context of linear viscoelasticity. By shear rates of the order of 10 s-1 the elastic nature of blood is negligible as evidenced by a merging of the oscillatory and steady flow viscosities. \textbf{However, there is a need to consider the finite viscoelastic behaviour of blood, if viscoelastic constitutive equations are used to model blood in the circulatory system}}

\textit{A number of \textbf{nonlinear viscoelastic constitutive models for blood are now available but because of their complexity we will avoid presenting the mathematical details here}, providing instead a summary of the relevant literature. One of the simplest rate type models accounting for the viscoelasticity of blood is the {\color{red}\textbf{Maxwell model}}
\begin{equation}
    \tau + \lambda_1\frac{\partial \tau}{\partial t} = 2\mu D
\end{equation}
where $\lambda_1$ is the relaxation time and $\frac{\partial \cdot}{\partial t}$ stands for the so-called convected derivative, a generalisation of the material time derivative, chosen so that $\frac{\partial \tau}{\partial t}$  is objective under a superposed rigid body motion and the resulting second-order tensor is symmetric[114] }                                   


\textit{A more general class of rate type models, includes the {\color{red}\textbf{Oldroyd-B}} models defined by}
\begin{equation}
     \tau + \lambda_1\frac{\partial \tau}{\partial t} = 2(\mu D+\lambda_2\frac{\partial D}{\partial t})
\end{equation})
\textit{where the material coefficient $\lambda_2$ denotes the retardation time (Verzoegerungszeit) and is such that  $0 \leq \lambda_2 < \lambda_1$. \textbf{the Oldroyd type fluids can be considered as Maxwell fluids with additional viscosity}. These models contain the previous model as a particular case.Thurston [129], was among the earliest to recognise the viscoelastic nature of blood and that the viscoelastic behaviour is less prominent with increasing shear rate. He proposed a generalised Maxwell model that was applicable to one dimensional flow simulations (see section 3) and observed later that, beyond a critical shear rate, the nonlinear behaviour is related to the microstructural changes that occur in blood [130]. Thurstons work was suggested to be more applicable to venous or low shear unhealthy blood flow than to arterial flows. Recently, a generalised Maxwell model related to the microstructure of blood, inspired on the behaviour of transient networks in polymers, and exhibiting shear thinning, viscoelasticity and thixotropy (defined below), has been derived }

\textit{Other viscoelastic constitutive models of differential type, suitable for describing blood rheology have been proposed in the recent literature. \textbf{The empirical three constant generalised Oldroyd -B model studied in [149] belongs to this class. It has been obtained by fitting experimental data in one dimensional flows and generalising such curve fits to three dimensions. This model captures the shear thinning behaviour of blood over a large range of shear rates but it has some limitations, since the relaxation times do not depend on the shear rate, which does not agree with experimental observations}. {\color{blue}The model developed by Anand and Rajagopal [9] in the general thermodynamic framework stated in [113] includes relaxation times depending on the shear rate and gives good agreement with experimental data in steady Poiseuille and oscillatory flow}}

\textit{Another important property of blood is its \textbf{thixotropic behaviour}, essentially due to the fact that the formation of the three-dimensional microstructure and the alignment of the RBCs are not instantaneous. Essentially, we refer to thixotropy as the dependence of the material properties on the time over which shear has been applied. This dependence is due to the finite time required for the build-up and breakdown of the 3D microstructure, elongation and recovery of RBCs and the formation and breakdown of layers of the aligned RBCs [13]}


\subsubsection*{Yield stress of blood} \textit{The behaviour of many fluids at low shear stress, including blood, has led researchers to believe in the existence of a critical value of stress below which the fluid will not flow. This critical stress level, called the yield value or yield, is typically treated as a constant material property of the fluid. [...] Reported values for the yield stress of blood have a great variation ranging from 0.002 to 0.40 dynes/c$m^2$ (...) \textbf{Rather than treating the yield stress as a constant, it should be considered as a function of time and linked to thixotropy, as later proposed by other researchers [89].}}

\textit{{\color{blue}\textbf{Yield stress models can be useful to model blood flow in low shear rate regions}}. Yield stress materials
require a finite shear stress $\tau_Y$ (the yield stress) to start flowing. A relatively simple, and physically relevant yield criterion is given by \begin{equation}
    \sqrt{\abs{\Pi}_{\tau}} = \tau_Y
\end{equation}
where $\sqrt{\abs{\Pi}_{\tau}}$ is the second invariant of the extra stress tensor, $\tau$  (see (1.5)). Therefore, for $ \sqrt{\abs{\Pi}_{\tau}} < \tau_Y,$ (1.9)
fluid will not flow!}


\textit{The most common yield stress model for blood is the {\color{red}\textbf{Casson model}} ([119]) which, in simple shear flow,
has the form\begin{align}
    \sqrt{\abs{\Pi}_{\tau}} < \tau_Y &\Rightarrow D=0 \\
    \sqrt{\abs{\Pi}_{\tau}} \geq \tau_Y &\Rightarrow  \begin{dcases*}
        D &= $\frac{1}{2\mu_n}(1- \frac{\sqrt{\tau_Y}}{ \sqrt{\abs{\Pi}_{\tau}}})^2\tau$ ,\\
        \tau &= $2(\sqrt{\mu_n}+\frac{\sqrt{\tau_Y}}{ \sqrt[4]{4\abs{\Pi}_{\tau}}})^2$
        \end{dcases*}
\end{align} \textbf{The Newtonian constitutive equation is a special case of (1.10) for $\tau_Y$ equal to zero, in which case, $\mu_N$ is the Newtonian viscosity. The Casson fluid behaves rigidly until (1.9) is satisfied, after which it displays a shear thinning behaviour.}} 


\textit{Other yield stress models like {\color{red}\textbf{Bingham}} or {\color{red}\textbf{Herschel-Bulkley models}} are also used for blood (see e.g.
[114]) as well as the {\color{red}\textbf{constitutive model developed by Quemada}} [112] using an approach, with the apparent viscosity $\mu$ given by
\begin{equation}
    \mu = \mu_f (1 -\frac{1}{2}\frac{k_0+k_{\infty}\sqrt{\Dot{\gamma}/\Dot{\gamma_C}}}{1+\sqrt{\Dot{\gamma}/\Dot{\gamma_C}}}\phi)^{-2}
\end{equation}
where $\mu_F, \phi \hspace{4pt }and \Dot{\gamma_C}$ are the viscosity of the suspending fluid, the volume concentration of the dispersed phase and a critical shear rate
}


\newpage
\subsection{Paper Recent advances in blood rheology: a review}

\subsubsection*{Introduction}
\textit{Experimentally, blood has been confirmed to demonstrate a variety of non-Newtonian rheological characteristics, including pseudoplasticity, viscoelasticity, and thixotropy. New rheological experiments and the development of more controlled experimental protocols on more extensive, broadly physiologically characterized, human blood samples demonstrate the sensitivity of aspects of hemorheology to several physiological factors. For example, at high shear rates the red blood cells elastically deform, imparting viscoelasticity, while at low shear rates, they form rouleaux structures that impart additional, thixotropic behavior.}

\textit{As the volume fraction of RBCs (hematocrit) by far exceeds that of any other component (see also Table 1), it is common to consider blood as a suspension of flexible RBCs in an otherwise Newtonian fluid, as will be discussed. Complex interactions between RBCs result in aggregates (rouleaux) that can grow to form a temporary network at low shear rates,5 which results in a significantly complex rheological behavior}


{\color{red} Consequently, hemorheology is characterized by a shear thinning viscosity, a general viscoelastic response to transient flow deformations and a non-zero yieldstress with associated
thixotropy . see, for example, a recent review2 and references therein}


\textit{Models derived from these advances range from simpler, generalized Newtonian, approaches, suitable mostly for steady-state shear-dominated flows, to thixotropic, elastoviscoplastic models, appropriate to study bloods complex transient rheological behavior. The emphasis of this review is continuum models to describe bulk blood rheology, although important connections to microscopic and multiscale models are also mentioned}

\subsubsection*{Hemorheology characteristics}
\textit{most obvious non-Newtonian characteristic of blood rheology is its shear thinning in steady shear flow (...) However, with time, people discovered that blood rheology exhibits additional more complex rheological characteristics, including viscoplasticity, viscoelasticity and thixotropy. }

\textit{TABLE with various quantities that can be used to characterize various aspects of hemorheology, such as various measures for limiting values (at zero and high shear rates) of blood viscosity, and its contribution from plasma and rouleaux, characterizing shear thinning, yield stress characterizing viscoplasticity, and three typical characteristic times useful to characterize two separate components of viscoelasticity (originating from the free RBC deformation, rouleaux aggregates) and thixotropy associated with rouleaux formation due to Brownian motion.}

\textit{In addition to a pronounced shear thinning behavior, the
presence of rouleaux in blood also gives rise to viscoelasticity and thixotropy. Due to the weak viscoelasticity of blood and a transition to nonlinear behavior at low strain amplitudes, the linear viscoelastic behavior of blood is typically difficult to measure. Thus, rheological techniques, such as large amplitude oscillatory shear (LAOS), which measure the thixotropy and viscoelasticity simultaneously, are more commonly used.27 At low shear rates, where the rouleaux are present, there is considerably more elasticity present in the sample; whereas, at high shear rates, the elasticity present is more representative of that of the isolated RBCs. The thixotropic behavior of blood arises from the structural evolution of the rouleaux which introduces a thixotropic time scale.}

 \textit{The complex rheological properties of blood are often
reported in an (overly) simplified manner by a few characteristic metrics. One such metric which governs the low shear behavior of blood is the yield stress. The yield stress of blood is typically very small, on the order of 1 mPa, and consequently difficult to measure, and can be easily missed if one is not careful to avoid the wall slip phenomenon.29 The most common method for determining}


\textit{The main physiological determinants of whole blood rheology
tend to be the hematocrit and the fibrinogen concentration, with the latter affecting the aggregation tendency of the rouleaux. Various authors suggest that the low shear behavior of blood is dependent on both these parameters, with aggregation occurring only above a certain critical hematocrit}


\subsubsection*{Experimental methods
Rheometry}...

\subsubsection*{Constitutive blood modeling}

{\color{blue}
\textbf{\textit{ From the four non-Newtonian characteristics of blood rheology,
namely shear thinning, yield stress, thixotropy and viscoelasticity, the first two can be easily captured through the use of a variety of generalized non-Newtonian equations}}}

\textit{Of these, a special mention here is the historic Casson model (eqn (2)),47 both because this was the workhorse model for describing blood viscosity,21,84 but primarily because it is the one for which we have the most extensive knowledge of the parametric dependence of the model parameters on important physiological parameters, such as the hematocrit and the fibrinogen concentration}

\textit{In its more natural form the Casson model is cast as a linear dependence of the square root of the shear stress with respect to the shear rate,47 giving rise to the following expression for the equivalent generalized shear viscosity}

\textit{
A major breakthrough in that respect has
been the validation of the previously proposed data collection
protocol and further elaboration on a systematic steady but
also transient shear experimental protocol by Horner et al.49 and
the availability of a set of well documented experimental data on
both human and several animal species blood see also sepa-rate section in this review on Comparative hemodynamics and
Hemorheology. This recent work has led to more accurate data
that not only advances our knowledge about non-Newtonian
blood rheology in transient flows, which is not captured of course
through inelastic, generalized Newtonian approaches, but also
suggests small, but important, corrections to the Casson model
predictions and the square root law even regarding the steady
state shear viscosity.13,52}




\subsubsection*{Thixotropic/elastoviscoplastic models for blood}
\textit{Generalized Newtonian models are computationally inexpensive to implement but cannot predict accurately transient (thixotropic/ hysteretic) changes in the viscosity which are relevant as blood flows naturally under pulsatile conditions. To better account for the transient effects, a viscoelastic model can be used}

\textit{One example of this is the Anand-Kwack-Masud (AKM) model which is a generalization of the Oldroyd-B model
\begin{align}
    \bm{\sigma} &= G \bm{B} + 2\eta_{\infty} \bm{D} \\
   \overset{\triangledown}{\bm{B}} &= (\frac{G}{\eta_0-\eta_{\infty}})(\frac{tr(\bm{B})}{3})^m(\bm{B}-\frac{3}{tr(\bm{B^{-1}}}\bm{I})
\end{align}
where $\sigma$ is the stress tensor, G is an elastic modulus, $\bm{B}$ is the left Cauchy-Green stretch tensor, $\bm{D}$  is rate of strain tensor, the superimposed inverted triangle denotes the upper-convected time derivative, m is a power law index, and I is the unit tensor. \textbf{Viscoelastic models similar to this approach are advantageous for modeling blood flow as they can account for elasticity in the sample and can predict non-shear components of the stress tensor. How- ever, they are typically unable to capture thixotropic effects that blood demonstrates associated with the structural evolution}
}

\textit{This parameter is then connected to a description describing the contribution to the stress by the rouleaux structures. Notable in this respect, in its pioneering description that unifies thixotropy and viscoelasti- city, is a model by Sun and De Kee that uses a generalized Maxwell viscoelastic model with viscoelastic parameters that depend on a structure parameter governed by a separate evolution equation}

\textit{One of the most successful thixotropic models for blood is
the Horner-Armstrong-Wagner-Beris (HAWB) model52 which uses this approach to describe the rouleaux contribution to the shear stress }


\newpage
\subsection{Paper A critical review on blood  flow in large  arteries; relevance to blood rheology, viscosity models,  and physiologic  conditions}


\textit{The  arterial blood flow in  the human body is typically  a
multiphase non-Newtonian pulsatile flow in a  tapered elastic duct with the terminal side and/or small  branches.The pulsatile flow is an unsteady  flow  in which resultant flow is  composed  of  a mean and  a periodically varying time-dependent component. Pulsatile flow  is responsible  for submissive effect  on  time-dependent viscoelastic and thixotropy behavior of  blood}


\subsubsection*{Blood composition and  structure}\textit{Blood composition  and structure play a vital role in blood  rheology. Blood consists of a suspension of  elastic particulate cells  in  a liquid known as plasma (..) Potential non-Newtonian properties of plasma  were considerable debate until the early 1960 (Zydney  et al., 1991), but  recent  studies have demonstrated  that plasma is  a Newtonian  fluid with  a viscosity which is a  function of temperature}

\textit{The aggregatable and  deformable nature of the red RBCs
plays significant roles in blood rheology. RBC aggregation causes a large increase in  viscosity at low shear rates. The size of RBC aggregation is a  function  of  RBC  concen- tration and shear rate (Zydney  et al., 1991).  The existence of aggregation also depends  on  the presence of fibrinogen and  globulin proteins in plasma (Fung,  1993). When  shear rate tends to  zero,  RBCs become  one big aggregate,  which then  behaves like  a solid. As  the shear  rate increases, RBCs aggregates  tend to be  broken  up and the  structure becomes a suspension of a  cluster of  RBCs aggregates in  plasma. These aggregates are  in turn formed  from smaller  units called rouleaux as shown. As the shear rate more  increases,  the average number of  RBCs  in each rouleaux  decreases. If the shear rate is larger than a certain  critical value,  the rouleaux are  broken up  into indi- vidual cells. At subcritical shear rates, the  RBCs  in each rouleaux maintain  their rest-state equilibrium  shape, bicon- cave discoid shape.  If the  shear rate is  supercritical, the RBCs  are dispersed in plasma separately  and tend to become  elongated  and line up with the streamlines  (see}

\textit{The  difference between the NP and NA curves indicates the effect of  cell aggregation, whereas that  between NA (suspended in albumin (leads to lack of other proteins in solution which prevents aggregation))  and HA (hardened RBCs in albumin) indicates the  effect of cell deformation. Aggregation of  red cells at low shear rates leads to  increase  of viscosity. Red cells  elongation and orientation  at high shear rates  lead to decrease of viscosity further}

\textit{the  blood  viscosity  is  dependent on 
the physiological flow conditions of  blood and the
blood composition  properties  such as hematocrit, temper- ature, shear rate,  cell aggregation,  cell shape, cell defor- mation and orientation. All of  these observed dependencies should be formulated  and  then systematically introduced into some  blood viscosity models for use in CFD analysis of blood flow. The review of the present viscosity models discussed in the following section however showed that; the  effects of  cell aggregation, cell shape,  cell deformation, and cell orientation have not  clearly reflected in the vis- cosity  models, although these  models considers  the effects of  hematocrit  or cell concentration, shear rate, and tem- perature.  However, numerical studies on  RBC behaviors give a  hope for  transferring available  knowledge of micro- scopic hemodynamic and hemorheological behaviors to a blood viscosity mode}

\subsubsection*{Blood viscosity  models}
\textit{Experimental investigations over many years showed that blood flow exhibits non-Newtonian behavior such  as shear thinning, thixotropy, viscoelasticity, and yield stress. Its  rheology is influenced by many factors  including plasma viscosity (Baskurt and Meiselman, 2003),  align- ment of  RBCs  (Baskurt  and Meiselman, 1977), level of RBC aggregation and deformability (Chien,  1970),  fibrin- ogen (Chien  et al, 1970), flow  geometry and size (Thurston and Henderson, 2006), rate of shear, hematocrit, male  or female, smoker  or non-smoker, temperature,  lipid loading, hypocaloric diet, cholesterol  level,  physical fitness  index (Cho  and Kensey, 1991), diabetes mellitus, arterial hyper- tension, sepsis (Meiselman and  Baskurt, 2006),  etc. The blood  viscosity  models in literature may however be dis- cussed in two main categories namely;\newline
\textbf{Newtonian viscosity models} and \textbf{non-Newtonian viscosity models}}


\subsubsection*{Newtonian viscosity models}\textit{The  blood behaves like a Newtonian  fluid when shear
rate over a limiting value. This apparent  limiting viscosity or shear rate is a function of  the blood  composition, and  is primarily modulated by hematocrit (...)  blood was modeled as a Newtonian fluid in some studies by accepting the viscosity of blood to be constant and equal to the blood viscosity (...) However, a considerable 
amount  of attempts has been developed for estimating Newtonian visscosity of whole blood as a  function of cell concentration - \textbf{concentration dependent
Newtonian models} (table given) - (...) n later devel- opments, interactions  between  particles have been  included by  adding a quadratic terms and the particle shape have been included}

\textit{Some of these models  were  extended to  the non-Newtonian mod- els by taking some model  parameters as shear dependent variables 
(Quemada, 1978;  Wildemuth and Williams,
1984; Snabre and  Mills, 1996)  and by  using the differential developing effective medium approach (Pal, 2003).}

\subsubsection*{Non-Newtonian viscosity models}
\textit{The instantaneous shear  rate  over  a cardiac cycle  varies from zero to  approximately 1000 s-1 in several large arteries  (Cho  and Kensey,  1991). Therefore, over a  cardiac cycle, there are  time periods where the blood exhibits shear thinning  behavior. In addition  to low  shear time  periods, low shear exists in some  regions such as near bifurcations, graft anastomoses, stenoses, and  aneurysm}



\textit{To model the  shear thinning properties of blood, a constitutive  equation is necessary to  define the relationship between  viscosity  and shear rate. The various non-Newtonian blood models have been  used to relate  the shear stress  and rate of deformation for  the blood flow  in large arterial  vessels. In addition  to shear thinning behavior, thixotropic and  elastic behaviors of blood have also been  taken into account  by various researchers. All of  these models can be  classified  into two categories as \textbf{time  independent}  and \textbf{time dependent flow behavior  models}}

\textit{The constants of the time  independent models were obtained by means of parameter  fitting on  experimental viscosity  data obtained  at certain shear  rates under steady  state  conditions}

\textbf{Experimental limitations:}
\textit{At this point it should  be noted  that; many existing vis-
cosity  data of whole blood were taken under assumption of Newtonian flow pattern in  the rheometers measurement field and ignored slip  effect.  In  addition  to these criteria, some data  have been fitted and  used  with neglecting the effect between physiologic blood temperature and medium temperature  of  measurement on CFD studies. All of these assumptions can cause to  errors!(...) Therefore,  apparent  shear rate  and viscosity need to corrections. Some  correlation methods for blood or  Casson  fluid  like blood  have been used  in the lit- erature (Janzen  et al., 2000; Joye, 2003; Zhang and  Kuang, 2000), but  they need to include the  effect  of hematocrit  or void fraction. Since different flow patterns may be observed at different  hematocrit values.}


\textit{The \textbf{slip effect} is  a common  feature  for all types of two- phase or  multiphase systems like  blood (Barnes, 1995).  It is  significant only if  the slip layer is sufficiently thick  or the viscosity of the slip  layer  is sufficiently low  (Coussot, 2005). Blood has the low viscosity of the continuous phase, the high concentration of the  suspensions,  and finally,  the relative large size of RBCs and its aggregates compared to usual wall roughness. Therefore, significant amount of slip is to be expected for blood under the appro- priate  conditions. Slipping of RBCs in  contact with the wall in  vivo and  in vitro  has been  observed in literatur (...) As a  result of  the shear induced migration of blood cells  away from wall boundaries, a cell-rich layer surrounded by a  cell-deplated plasma  layer  at the wall occurs. et al., 1992). The interface between these layers is a  rough surface  due  to the  presence of the  RBCs, and the protrusion of blood cells into the plasma layer  may give additional energy  dissipation (Sharan and Popel, 2001). Narrow marginal layer of plasma has a  lower vis- cosity than the rest  of the fluid and  serve as a  lubricant so that slip occurs. The slip effects and the effects of fibrinogen level and hematocrit  on it  at low shear  rates were studied (Picart  et al., 1998a; Picart  et al., 1998b)  and found that; migrational and  slip effects are more pronounced as shear rate decreased, fibrinogen concentration is raised, and hematocrit is lowered. Aggregation behavior of RBCs at low shear rates and increase  of the fibrinogen level could give rise to effective particle size, and lower hematocrit could give rise to the slip layer thickness. As a result, the slip  effect can  cause to experimental errors on viscosity data of  whole blood if not eliminate or taken into accoun (\textbf{slip effect leads to erros})}

\textit{Another  experimental limitation is  that the viscosity
models are based  on  experimental data obtaining the result- ing  shear stress  response to  shear  rate after an acceptable amount of time has passed to allow rouleaux formation, although  the time between  consecutive strokes  is not suf- ficiently long  for the segregated RBC  to reaggregate}



\subsubsection*{Time independent  viscosity models}
\textit{show shear  thinning behavior and must  meet  the following  model 
requirements: There are three  distinct
regions  for apparent blood viscosity:
\begin{enumerate}
    \item lower Newtonian region  (low shear rate  constant viscosity,  $\mu_0$)
     \item an  upper Newtonian region (high shear rate  constant  viscosity,  $\mu_{\infty}$),
     \item and  a middle region  where the apparent viscosity is decreasing with increasing shear  rate,$\frac{\partial \mu}{\partial \gamma} < 0$ 
\end{enumerate}
\textbf{Power  law equation} is a suitable model for the middle region. However, it does not  describe the low and high shear rate regions \newline
\textbf{Herschel-Bulkley  model } extends the power  law model to include  the yield  stress, $\tau_{\gamma}$ \newline The other models
which include yield stress are  also shown in Table 3
}

\textit{The models  that  have limitation at low and  high shear rates can easily figure out when shear rate  in models  are taken as zero or infinity (whole table with all the models zero or infinity. Casson,  Walburn-Schneck, and  Weaver models include  the effect  of  the RBC concentration,  
Walburn-Schneck model also incorporates fibrinogen and globulins proteins, TPMA. A log-log plot of the apparent blood viscosity as  a function of shear rate can be also used to figure out how much  is closely matched each model and model parameters with healthy experimental blood vis- cosity data.}

\textit{The study on 11  viscosity models of Easthope and Brooks (1980) 
concluded that  the Walburn and Schneck model is in well agreement in the shear rate range of 0.03-120 s-1. Sugiura (1988) showed  that viscosity
model in that shear  rate should be  Weaver  model ... different results}

\textbf{TABLE WITH MANY TIME INDEPENDENT MODELS}


\subsubsection*{Time dependent viscosity  models}
\textit{Time-dependent models are used to describe  thixotropic
and  viscoelastic behavior. Thixotropy  is  a special condition of pseudo plasticity with or without yield stress, where the apparent viscosity also decreases when the fluid is subjected to a constant  shear rate. 
 Blood is a  concentrated sus- pension of cells.  Most  of concentrated suspension system exhibits thixotropic behavior. Experimental results of whole human blood  have  demonstrated that blood is a exhibits thixotropic behavior. ...the hematocrit, gender and age  are factors  affecting  the blood thixotropy}
 
 \textit{Chen  et al., 1 991;  Huang  et al., 1987b). Huang developed a generalized  rheological equation for thixotropic fluids. Huang model parameters were calculated from experimental data by non-linear parameter estimation technique for  apparently healthy  human subjects}

\textbf{TABLE WITH 4 TIME-DEPENDENT THIXOTROPIC MODELS}
Huang model, Weltman Model, Tiu-Boger Model, Rosen Model 


\textit{n literature, viscous models obtained at steady state  con- ditions were used under pulsatile flow conditions. These models have  provided  a great deal  of insight  into the  non- Newtonian viscous behavior of blood. However,  blood exhibits both viscous and elastic  properties under pulsatile flow }

\textbf{
\textit{Viscosity  is  an assessment  of the rate of energy dissipation due to cell deformation and sliding, while elasticity is an assessment of  the elastic  storage of  energy  primarily due to kinetic deformability of the RBCs}}


\textit{There are  two dimensionless numbers to measure of the tendency of a material to  appear either viscous or  elastic; the  Deb-orah number and the  Weissenberg number. The Deborah number is the  ratio of the relaxation time to  the characteristic time of the flow $De= \Theta_r/ T_c$, with The relaxation time can  be defined as  a ratio of the viscosity to elatisity $\Theta_r=\mu/G$. Three ranges  of Deborah  number  are  identified as that; if
De $<<1$ ,  the material is  viscous, if De $\approx1$ ,  the material will act viscoelastically, and if De $>>1$, the material is elastic (Steffe, 1996).}


\textit{The Weissenberg number is  defined as the ratio of  the characteristic time of fluid to  a characteristic convective time-scale of the flow or the characteristic shear rate times the relaxation time $We = \Theta_r \Dot{\gamma_c}$}

\textit{Blood is slightly  viscoelastic, and  its  effect was  ignored in
most  of the CFD studies. At low shear rates, RBCs aggregate are solid-like bodies, and  has ability to store elastic energy.  Its value remains constant  for shear rates up  to 1 s-1
 (Thurston, 1979).  At  high shear rates, its  effect is less  prominent because the RBCs behave fluid-like  bodies (Schmid-Schoenbein and Wells, 1969), and  lose this ability  (Anand and Rajagopal, 2004). Viscoelastic models can be suitable  for  blood flow under certain flow conditions  especially at  low shear  stress. It also grows in importance when flow is oscillatory (Rojas,}
 
 \textit{The extra stress tensor is used to characterize the viscoelastic stress in viscoelastic models. The total stress tensor (Cauchy stress tensor) is defined in terms of the pressure (P) and the  extra stress tensor (S). $\bm{\sigma}= -P\bm{I+ \bm{S}}$}

Different models e.g. Oldroyd-B Model, Yeleswarapu Model, Generalized Oldroyd-B Model (Rajagopal and Srin- ivasa, 2000), Generalized Maxwell Model (Rajagopal and Srin- ivasa, 2000), Generalized Oldroyd-B Model (Anand and Rajago- pal, 2004), Generalized Maxwell Model (Owens,  2006):


\subsubsection*{Yield stress of blood}
\textit{The yield  stress of blood can  be taken  into  account in  the
regions of low shear rates like thixotropic and viscoelastic behavior of blood.  It can supply better  understanding  of aggregation  and  deformation of blood cells. The  relation of cell deformation and aggregation with yield stress was described as}

\subsubsection*{Physiological conditons}
\textit{The cyclic nature  of the  heart  pump creates  pulsatile con-
ditions in arteries, and therefore blood flow and  pressure are unsteady. The pulsations of  flow  are damped in the small vessels, and the  flow is so effectively  steady in  the capillaries and the veins.  The arteries are not rigid tubes. It adapts to varying flow and pressure  conditions  by  enlarg- ing or shrinking. All of these  physiologic conditions and others cause difficulties to  render simulations  both con- ceptually and computationally tractable. As a result, most CFD  models of arterial  hemodynamics need  to make  the simplifying assumptions of  rigid walls, steady flow, fully developed  inlet velocities, Newtonian rheology,  normal and  periodic flow conditions, ignoring of small side or ter- minal  branches, using  of  idealized or averaged artery  mod- els, and using of in vitro  experiments to validate CFD models (Steinman, 2002; Steinman,  2004; Steinman and Taylor, 2005)}


 \subsubsection*{Conclusion}
\textit{Although there has been  the considerable amount of viscosity models in blood  rheology, none of them  has been commonly agreed upon or used. None of the models is fully expressing the effects of extremely complicated nature of blood  rheology and  its  depen- dence of many factors. The groovy method used for obtaining the existing viscosity models is  the para- metric curve fitting on experimental apparent  viscos- ity versus shear rate  data,  although the  blood con- stitutive parameters in these models have been found to be  related to internal  structures of the blood such  as RBC aggregation, RBC deformability,  etc. The dependency of blood  viscosity on each rheological property of the  blood should  be investigated and  clar- ified separately  by  means of  isolating the effects of remaining rheological properties.}


\textit{The  re-aggregation and  viscoelastic behavior of  blood detected in  realistic pulsatile flows are  much  different from  the corresponding behavior  in steady  flow. The blood viscosity models fitted on the  data measured in the rheometer under the steady flow conditions are so not suitable for the  analysis of  realistic  pulsatile blood flow. Observed blood apparent  viscosity in  the rhe- ometers under the assumption of Newtonian flow pat- tern is also not suitable and the hematocrit value of tested blood also affects this pattern. The assumption of Newtonian pattern and  neglecting the  hematocrit effect on  it can cause  errors. The  slip effect  in  the rhe- ometers  should also be  prevented throughout  the mea- surements.}

\textit{Thixotropy must be  considered  in the regions  of  a substantial RBC residual  time  to allow the  separated RBC to  re-aggregate  such as aneurysms and  recir- culation zones}

\textit{Viscoelasticity  and thixotropy  have generally accept- ed  as  less important for high shear  regions. Even  so, there is more need  to develop in this area  for  low shear  regions. Recent  increases in  study on viscoelas- tic models  and its application in literature may be an evidence of this.}

\newpage
\section{Derivation of Navier-Stokes Equation}


\subsection{Cardiovascular mathematics Chapter 3}

\subsubsection*{Momentum equation}
\textit{The conservation of (linear) momentum is in fact the well known \textbf{Newtons law}. The rate of change of the momentum of a material domain $V(t)$, given by $\int_{V(t)} \rho \bf{u} d\bf{x} $ equals the resultant of the external forces acting on it, that is \begin{equation}
   \frac{d}{dt} \int_{V(t)} \rho \bf{u} d\bf{x} = \bm{F} = \bm{F_v} + \bm{F_s}
\end{equation}
The force F is the composition of two terms:\textbf{ a volume force $F_v$,and a surface force $F_s$}. The former acts on each particle of V(t) (like the force of gravity) and is expressed as the integral of the density times a specific force (i.e. force per unit of weight) f which has the dimension of an acceleration,}

\textit{The surface force is instead responsible for the mutual interaction between the material contained in V(t) and the exterior, through the boundary $\partial$V(t). More precisely, $F_s$ is equal to the surface integral of the so called Cauchy
stress $\bm{T}(\bm{x}, t, \bm{n})$, which has the dimension of force per unit area, [t]= N/$m^2$,that is
\begin{equation}
    F_s =  \int_{\partial V(t)} \bm{T} d\bf{\gamma}
\end{equation}
It was indeed Cauchy who also postulated that $\bm{T}$ can be computed by
applying to the normal $\bm{n}$ of $\partial V(t)$ a symmetric second-order tensor
\begin{equation}
    \bm{\sigma} : \Omega(t) \longrightarrow \mathrm{R}^{3x3}
\end{equation}
called the \textbf{Cauchy stress tensor} $\bm{\sigma}(\bm{x},t)$, i.e.
\begin{equation}
    \bm{T} = \bm{\sigma}\bm{n} \hspace{5pt} on \hspace{5pt}  \partial V(t)
\end{equation}
}

{\color{red}\textbf{The momentum conservation law can then be expressed by the following}}
equation \begin{equation}
     \frac{d}{dt} \int_{V(t)} \rho \bf{u} d\bf{x} = \int_{V(t)} \rho \bf{f} d\bf{x} + \int_{V(t)} div \bf{\sigma} d\bf{x}
\end{equation}

\textit{To obtain the last equality we have used the divergence theorem. Finally, by exploiting the Reynolds transport formula: \newline
  Let $V(t)$ be a material domain, i.e. $V(t)= \bm{x} : \bm{x}= \bm{\Tilde{\phi}}(\bm{\Tilde{x}},t), \hspace{5pt} \bm{\Tilde{x}} \in  \bm{\Tilde{V}},$and $f$ a   continuously differentiable field, Then
\begin{align}
 \frac{d}{dt}\int_{V(t)} f \hspace{2pt}d\bm{x} = \int_{V(t)} ( \frac{\partial f}{\partial t} + div(f\bm{u}) )\hspace{2pt} d\bm{x}
\end{align}
}

\textit{So finally we obtain: Momentum conservation in quasi-linear form
\begin{equation}
\rho  \frac{\partial \bm{u}}{\partial t} + \rho (\bm{u} \cdot \nabla) \bm{u} = div \bm{\sigma} + \rho\bm{f}, \hspace{5pt} in \hspace{5pt} \Omega(t), \hspace{5pt}t>0
\end{equation}
and in conservation form
\begin{equation}
 \frac{\rho \partial \bm{u}}{\partial t} + div(\rho \bm{u} \cross \bm{u}-\bm{\sigma}) =\rho\bm{f}, \hspace{5pt} in \hspace{5pt} \Omega(t), \hspace{5pt}t>0
\end{equation}
The transport term $(\bm{u} \cdot \nabla) \bm{u}$ is a nonlinear term and rises the complexity to solve equation
}

\textit{At each point of the boundary of a material domain V(t) the Cauchy
stress $\bm{T}$ can be decomposed into its components normal and tangential to the surface, given respectively by\begin{equation}
    T_n = \bm{T} \cdot \bm{n} = (\bm{\sigma} \bm{n}) \cdot \bm{n}, \hspace{10pt }\bm{T}_t = \bm{T}- T_n\bm{n}
\end{equation}
 The latter is indeed a vector laying on the tangential plane and is called
the \bf{shear stress vector}.
\textbf{It is an important parameter in haemodynamics since the endothelium cells are very sensitive to the shear stress at the vessel walls}. The equations
have been here written in Eulerian formulation
}

\subsubsection*{Fluids and Solids - Definition of Cauchy stress tensor!} \textit{We need now to make precise how the Cauchy stress tensor is linked to the kinematics. It is indeed at this point where the behaviour of solids and fluids diverges- As solids react to deformations, the Cauchy stress must depend on the deformation gradient $\bm{\Tilde{F}}$. 
Fluids instead can adapt to a deformation, as a fluid can fill freely any
arbitrary shape. Yet it takes time to fill it. And oil takes more time than water. It means that fluids react mechanically not to the deformation itself but to its rate. More precisely, the relevant quantity is here the strain rate tensor $\bm{D}$ defined in (2.2) of Chapter 2, and whose dimensions are [$\bm{D}$ ]= $s^{-1}$. Componentwise, the strain rate reads}(Chapter 3) 
\begin{equation}
    D_{ij} = \frac{1}{2} \large( \frac{\partial u_i}{\partial x_j} + \frac{\partial u_j}{\partial x_i} \large) \hspace{10pt} i,j=1,...3
\end{equation}
As in one of the videos the strain rate tensor corrsponds to the symmetric part of velocity gradient tensor
\textit{\textbf{In a fluid then  the cauchy stress tensor is a function of $\bm{D}$, while it is independent of  deformation gradient $\bm{F}$. A consequence is that the reference configuration is a concept useful for the derivation of the equations, yet it does not play any particular role for a fluid. Intermediate behaviours, like that of visco-elastic fluids, for instance, are possible; they will be addressed in detail in Chapter 6.} The relation between the Cauchy stress tensor $\bm{\sigma}$ and the kinematic quantities is called constitutive relation, or constitutive law, and is a characteristic of the type of material under consideration. To be physically correct, a constitutive relation must obey certain rules, like the principle of material frame indifference [512] which states that the relation should be invariant under a change of frame of reference.}(Chapter 3)

\subsubsection*{Newtonian and Non-Newtonian}
\textit{ In a Newtonian incompressible fluid (this is a limitation usually accepted for blood flow in large arteries), the Cauchy stress tensor depends
linearly on the strain rate. More precisely, we have \begin{equation}
    \bm{\sigma} = \bm{\sigma}(\grad\bm{u},P)= -P\bm{I} +2\mu\bm{D}(\grad\bm{u})= -P\bm{I} + \mu(\grad\bm{u}+\grad\bm{u}^T)
\end{equation}
where P is the pressure, I is the identity matrix, $\mu$ is the dynamic viscosity of the fluid and is a positive quantity. The term $2\mu\bm{D}(\grad\bm{u})$is the viscous stress component of the stress tensor. We
have that [P]= N/$m^2$ and [$\mu$]= kg/ms. The viscosity may vary, for example it may depend on the fluid temperature. The assumption of Newtonian fluid, however, implies that $\mu$ is independent of kinematic quantities. \textbf{Simple models for non-Newtonian fluids, often used for blood flow simulations, express the viscosity as function of the strain rate, that is $\mu = \mu(\bm{D}(\bm{u}))$}. }

\newpage
\section{General}

\subsection*{Overview Non-Newtonian Models
}


\newpage
\subsection*{Resume}
\begin{enumerate}
    \item In certain regions (e.g. low shear regions: e.g. near bifurcations, graft anastomoses, stenoses, and  aneurysm) the Non-Newtonian behaviour of blood shound not be neglected and integrated into the model equation
    \item Hemorheology is characterized by 
four non-Newtonian characteristics 
\begin{enumerate}
 \item a \textbf{shear thinning viscosity}
 \item a general \textbf{viscoelastic} response to transient flow deformations and 
 \item a \textbf{non-zero yieldstress} with
 \item associated \textbf{thixotropy}
\end{enumerate}
    \item Starting point for the flow analysis is in a macroscopic viewpoint the continuity equation and momentum equation
    \item In order to account for Non-Newtonian effects, we can\begin{enumerate}
        \item use linear elastic constitutive equation to close system and make the viscosity dependent of the deformation rate tensor (Generalized non-Newtonian equations e.g. Power law)
        \item use a (nonlinear) viscoelastic constitutive equation $\Rightarrow$ more complex constitutive equations must be solved simultaneously along with the equations of conservation of mass and momentum.
    \end{enumerate}  
\item  Generalized non-Newtonian equations can capture a) shear thinning and c) yield stress. They are computationally inexpensive
to implement but cannot predict accurately transient (thixotropic/
hysteretic) changes in the viscosity which are relevant as blood
flows naturally under pulsatile conditions.
\item To better account for
the transient effects, a viscoelastic model can be used. 
Viscoelastic models can account for elasticity in the sample
and can predict non-shear components of the stress tensor. However, they are typically unable to capture thixotropic effects that
blood demonstrates associated with the structural evolution.
\item Thixotropic models to
describe the transient rheology of blood
\end{enumerate}


\newpage
\subsection*{Questions}
\begin{itemize}
    \item  \textit{"Capillaries, the smallest blood vessels, are very different in their properties and connectivity. Capillaries can be as small as 5 micrometer diameter, which is considerable smaller than the largest diameter of a red blood cell." (Preface Book Cardiovascular mathematics)} \newline
    \textit{"Things get even more complex in the smallest capillaries, since here the size of a red blood cell becomes comparable to that of the vessel and the continuum hypothesis may become questionable" (Chapter 2 Book Cardiovascular mathematics)}\newline
    but \textit{"Blood flow on scales larger than 300$\mu$m is consistently modeled as a continuous fluid, as it is computationally more convenient because models no longer include the individual cell dynamics [20,24,25]"}(A heterogeneous multi-scale model for blood flow)
\newline
{\color{red}What is the relevant diamter which we want to examine?}\newline \textit{Possible locations where the non-Newtonian behaviour will be significant include segments of the venous system and stable vortices downstream of some stenoses and in the sacs of some aneurysms} \textit{In summary, we can conclude that blood is generally a non-Newtonian fluid, which can however be
regarded as a Newtonian fluid to model blood flow in arteries with diameters larger than 100 $\mu$m where measurements of the apparent viscosity show that it ranges from 0.003 to 0.004 Pa.s and the typical Reynolds number is about 0.5 (see Methods of Blood Flow Modelling)}.\newline
{\color{red} What are our regions of interest?} \newline \textit{"A simple model of a straight rigid blood vessel with unsteady periodic flow is considered. A numerical solution that considers the fully coupled Navier Stokes and energy equations is used for the simulations"}(A review on rheology of non-Newtonian properties of blood) \newline
{\color{red}Heat transfer?}
{\color{red} Are we looking for these kind of phenomena?}
    \item Rather genereal question:((\textit{"At a macroscopic level, the arterial wall is a complex multi-layer structure which deforms under the action of blood flow. Even though sophisticated constitutive equations have been proposed for the structural behaviour of the vessel wall, its elastic characteristics in vivo are still very difficult to determine and are usually inferred from pulse propagation data. The modelling of the interaction between blood flow and the vessel wall mechanics needs algorithms which correctly describe the energy transfer between them to accurately represent wave propagation phenomena." (Preface Book Cardiovascular mathematics)} \newline
    {\color{red}How far is the development of Trilions FSI-code?} ))
\end{itemize}

\end{document}